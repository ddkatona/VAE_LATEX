%\documentclass[12pt,a4paper,oneside]{report}             % Single-side
\documentclass[12pt,a4paper,twoside,openright]{report}  % Duplex

%\PassOptionsToPackage{chapternumber=Huordinal}{magyar.ldf}
\usepackage{t1enc}
\usepackage[utf8]{inputenc}
\usepackage{amsmath}
\usepackage{amssymb}
\usepackage{enumerate}
\usepackage[thmmarks]{ntheorem}
\usepackage{graphics}
\usepackage{epsfig}
\usepackage{listings}
\usepackage{color}
%\usepackage{fancyhdr}
\usepackage{lastpage}
\usepackage{anysize}
\usepackage[magyar]{babel}
\usepackage{sectsty}
\usepackage{setspace}  % Ettol a tablazatok, abrak, labjegyzetek maradnak 1-es sorkozzel!
\usepackage[hang]{caption}
\usepackage{hyperref}
%\usepackage{natbib}
%\usepackage[hidelinks]{hyperref}
\usepackage{array}
% Default fixed font does not support bold face
\DeclareFixedFont{\ttb}{T1}{txtt}{bx}{n}{12} % for bold
\DeclareFixedFont{\ttm}{T1}{txtt}{m}{n}{12}  % for normal

% Custom colors
\usepackage{color}
\definecolor{deepblue}{rgb}{0,0,0.5}
\definecolor{deepred}{rgb}{0.6,0,0}
\definecolor{deepgreen}{rgb}{0,0.5,0}

\usepackage{listings}
% Python style for highlighting
\newcommand\pythonstyle{\lstset{
language=Python,
basicstyle=\ttm,
otherkeywords={self},             % Add keywords here
keywordstyle=\ttb\color{deepblue},
emph={MyClass,__init__},          % Custom highlighting
emphstyle=\ttb\color{deepred},    % Custom highlighting style
stringstyle=\color{deepgreen},
frame=tb,                         % Any extra options here
showstringspaces=false            % 
}}


% Python environment
\lstnewenvironment{python}[1][]
{
\pythonstyle
\lstset{#1}
}
{}

% Python for external files
\newcommand\pythonexternal[2][]{{
\pythonstyle
\lstinputlisting[#1]{#2}}}

% Python for inline
\newcommand\pythoninline[1]{{\pythonstyle\lstinline!#1!}}





%--------------------------------------------------------------------------------------
% Main variables
%--------------------------------------------------------------------------------------
\newcommand{\vikszerzo}{Katona Dániel}
\newcommand{\viktemavezeto}{Braun Patrik János}
\newcommand{\viktemavezetoaffil}{BME AUT}
\newcommand{\vikkonzulens}{dr. Varga Dániel}
\newcommand{\vikkonzulensaffil}{MTA Rényi Intézet}

\newcommand{\vikcim}{Auto-encoder alapú generatív modellek kiértékelése}
\newcommand{\viktanszek}{Automatizálási és Alkalmazott Informatikai Tanszék}
\newcommand{\vikdoktipus}{Szakdolgozat}
\newcommand{\vikdepartmentr}{??????}

%--------------------------------------------------------------------------------------
% Page layout setup
%--------------------------------------------------------------------------------------
% we need to redefine the pagestyle plain
% another possibility is to use the body of this command without \fancypagestyle
% and use \pagestyle{fancy} but in that case the special pages
% (like the ToC, the References, and the Chapter pages)remain in plane style

\pagestyle{plain}
%\setlength{\parindent}{0pt} % áttekinthetõbb, angol nyelvû dokumentumokban jellemzõ
%\setlength{\parskip}{8pt plus 3pt minus 3pt} % áttekinthetõbb, angol nyelvû dokumentumokban jellemzõ
\setlength{\parindent}{12pt} % magyar nyelvû dokumentumokban jellemzõ
\setlength{\parskip}{0pt}    % magyar nyelvû dokumentumokban jellemzõ

\marginsize{35mm}{25mm}{15mm}{15mm} % anysize package
\setcounter{secnumdepth}{0}
\sectionfont{\large\upshape\bfseries}
\setcounter{secnumdepth}{2}
\singlespacing
\frenchspacing

%--------------------------------------------------------------------------------------
%	Setup hyperref package
%--------------------------------------------------------------------------------------
\hypersetup{
    bookmarks=true,            % show bookmarks bar?
    unicode=false,             % non-Latin characters in Acrobat’s bookmarks
    pdftitle={\vikcim},        % title
    pdfauthor={\vikszerzo},    % author
    pdfsubject={\vikdoktipus}, % subject of the document
    pdfcreator={\vikszerzo},   % creator of the document
    pdfproducer={Producer},    % producer of the document
    pdfkeywords={keywords},    % list of keywords
    pdfnewwindow=true,         % links in new window
    colorlinks=true,           % false: boxed links; true: colored links
    linkcolor=black,           % color of internal links
    citecolor=black,           % color of links to bibliography
    filecolor=black,           % color of file links
    urlcolor=black             % color of external links
}

%--------------------------------------------------------------------------------------
% Set up listings
%--------------------------------------------------------------------------------------
\lstset{
	basicstyle=\scriptsize\ttfamily, % print whole listing small
	keywordstyle=\color{black}\bfseries\underbar, % underlined bold black keywords
	identifierstyle=, 					% nothing happens
	commentstyle=\color{white}, % white comments
	stringstyle=\scriptsize\sffamily, 			% typewriter type for strings
	showstringspaces=false,     % no special string spaces
	aboveskip=3pt,
	belowskip=3pt,
	columns=fixed,
	backgroundcolor=\color{lightgray},
} 		
\def\lstlistingname{lista}	

%--------------------------------------------------------------------------------------
%	Some new commands and declarations
%--------------------------------------------------------------------------------------
\newcommand{\code}[1]{{\upshape\ttfamily\scriptsize\indent #1}}

% define references
\newcommand{\figref}[1]{\ref{fig:#1}.}
\renewcommand{\eqref}[1]{(\ref{eq:#1})}
\newcommand{\listref}[1]{\ref{listing:#1}.}
\newcommand{\sectref}[1]{\ref{sect:#1}}
\newcommand{\tabref}[1]{\ref{tab:#1}.}

\DeclareMathOperator*{\argmax}{arg\,max}
%\DeclareMathOperator*[1]{\floor}{arg\,max}
\DeclareMathOperator{\sign}{sgn}
\DeclareMathOperator{\rot}{rot}
\definecolor{lightgray}{rgb}{0.95,0.95,0.95}

\author{\vikszerzo}
\title{\viktitle}
%\includeonly{
	%guideline,%
	%project,%
	%titlepage,%
	%declaration,%
	%abstract,%
	%introduction,%
	%dolgozat,
	%chapter1,%
	%chapter2,%
	%chapter3,%
	%acknowledgement,%
	%appendices,%
%}
%--------------------------------------------------------------------------------------
%	Setup captions
%--------------------------------------------------------------------------------------
\captionsetup[figure]{
%labelsep=none,
%font={footnotesize,it},
%justification=justified,
width=.75\textwidth,
aboveskip=10pt}

\renewcommand{\captionlabelfont}{\small\bf}
\renewcommand{\captionfont}{\footnotesize\it}

%\usepackage[hyperref=true,backend=biber,dateabbrev=false]{biblatex}
%\addbibresource{mybib.bib}
%\usepackage{xcolor}
%\colorlet{doic}{green!50!black}
%\colorlet{arxivc}{pink!50!black}
%\usepackage[hyperref=true,backend=biber,style=numeric]{biblatex}
%%\usepackage[hidelinks]{hyperref}
%%\urlstyle{sf}
%
%% ARXIV links       
%\def\arxivtourl#1{http://arxiv.org/abs/#1}      
%\DeclareFieldFormat{eprint:arxiv}{%
    %\ifhyperref
        %{\color{arxivc}\href{\arxivtourl#1}{[arxiv:\nolinkurl{#1}]}}
        %{\nolinkurl{#1}}}

%--------------------------------------------------------------------------------------
% Table of contents and the main text
%--------------------------------------------------------------------------------------
\begin{document}
\singlespacing
%\include{guideline}
\include{project}

\pagenumbering{arabic}
\onehalfspacing
\include{titlepage}
\tableofcontents\vfill
\include{declaration}
%----------------------------------------------------------------------------
% Abstract in hungarian
%----------------------------------------------------------------------------
\chapter*{Kivonat}\addcontentsline{toc}{chapter}{Kivonat}

A dolgozatban bemutatjuk a manapság legsikeresebb generatív modelleket, különös hangsúlyt fektetve az autoencoder alapú VAE, illetve $\beta$-VAE architektúrákra, melyek generatív modellként is alkalmazhatóak, de a reprezentáció-tanulásban is komoly jelentőségre tettek szert. Munkánk során különböző metrikák megalkotásával formalizáljuk és számszerűsítjük, hogy ezen modellek reprezentációi mennyire felelnek meg az ilyen reprezentációk iránt támasztott informális követelményeknek. Ehhez egy szintetikus adathalmazt, a Dsprite-ot használjuk.

A vizsgálathoz szükséges programokat Kerasban, Python nyelven a numpy/scikit-learn/tensorflow/matplotlib könyvtárakra építve implementáljuk, modelljeiket pedig nagy teljesítményű grafikus kártyákon futtatjuk. 

Az így felálló keretrendszerrel számos különböző VAE alapú modellt hozunk létre és ezeket reprezentáció szempontjából, saját metrikáink alapján hasonlítjuk össze egymással. Különös hangsúlyt fektetünk a $\beta$-VAE hálók $\beta$ hiperparaméterének rep\-re\-zen\-táció-hatására, illetve a rekonstrukciós képesség és a látens tér struktúrájának egymáshoz való viszonyára. A reprezentációk személtetéséhez többféle ábrát is generálunk minden modellhez, illetve modell halmazhoz. 

\vfill

%----------------------------------------------------------------------------
% Abstract in english
%----------------------------------------------------------------------------
\chapter*{Abstract}\addcontentsline{toc}{chapter}{Abstract}

The paper presents the most successful generative models nowadays, with particular emphasis on autoencoder-based VAE and $\beta$-VAE architectures, which can be used as a generative model, but have also gained significant importance in representational learning. Through our work, we create and quantify how the representations of these models meet the informal requirements for such representations by creating different metrics. For this, we use a synthetic data set, Dsprite.

The programs for testing are implemented in Keras, Python built on the libraries of numpy/scikit-learn/tensorflow/matplotlib, and we run their models on high-performance graphics cards.

With this framework, we create a variety of VAE-based models and compare their representations to each other, based on our own metrics. Particular emphasis is placed on the representation effect of the $\beta$ hyperparameters of $\beta$-VAE nets, and on the relationship between the  capability of reconstruction and latent space. For the visialization of representations, we generate several figures for each model and model set.
\vfill


%\include{introduction}
\documentclass[12pt]{article}
\usepackage[utf8]{inputenc}
\usepackage{graphicx}
\usepackage{array}  
%\usepackage{listings}
\renewcommand{\figurename}{Ábra.}
% Default fixed font does not support bold face
\DeclareFixedFont{\ttb}{T1}{txtt}{bx}{n}{12} % for bold
\DeclareFixedFont{\ttm}{T1}{txtt}{m}{n}{12}  % for normal

% Custom colors
\usepackage{color}
\definecolor{deepblue}{rgb}{0,0,0.5}
\definecolor{deepred}{rgb}{0.6,0,0}
\definecolor{deepgreen}{rgb}{0,0.5,0}

\usepackage{listings}

% Python style for highlighting
\newcommand\pythonstyle{\lstset{
language=Python,
basicstyle=\ttm,
otherkeywords={self},             % Add keywords here
keywordstyle=\ttb\color{deepblue},
emph={MyClass,__init__},          % Custom highlighting
emphstyle=\ttb\color{deepred},    % Custom highlighting style
stringstyle=\color{deepgreen},
frame=tb,                         % Any extra options here
showstringspaces=false            % 
}}


% Python environment
\lstnewenvironment{python}[1][]
{
\pythonstyle
\lstset{#1}
}
{}

% Python for external files
\newcommand\pythonexternal[2][]{{
\pythonstyle
\lstinputlisting[#1]{#2}}}

% Python for inline
\newcommand\pythoninline[1]{{\pythonstyle\lstinline!#1!}}
\begin{document}


\section{Bevezetés}

\section{Alapok}


\subsection{Neurális hálók}

A neurális hálók biológiai indíttatású programok, amelyek a biológiai neurális hálózat néhány hasznos tulajdonságát modellezik.

\subsubsection{Alapvető felépítése}

A neurális hálók mindössze bonyolult, sok paraméteres, többdimenziós függvények, amelyek egy $n$ dimenziós inputhoz $k$ dimenziós outputot határoznak meg.

Például egy emberhez, pontosabban annak adataihoz hozzárendelnek egy betegséget olyan módon, hogy a négy output közül az veszi fel az $1$ értéket amely betegség a függvény “válasza”, az összes többi (betegséghez tartozó) output pedig $0$. A hálózat tehát meg tudja becsülni, hogy az illető milyen betegségben szenved.

\begin{figure}[h!]
  \includegraphics[width=\linewidth]{fgv.png}
  \caption{Általános nézet}
\end{figure}

Ez így elég egyszerűen hangzik, mivel még nem tárgyaltunk arról, hogy ez a függvény hogyan működik, és honnan tud néhány adatból betegségre vonatkozó következtetéseket levonni. A kérdés tehát, hogy hogyan határozzuk meg ezt a függvényt. Legyen a függvény egy neurális háló:

\begin{figure}[h!]
  \includegraphics[width=\linewidth]{fgv_network.png}
  \caption{Konkrét neurális háló}
\end{figure}


A neurális háló tulajdonképpen egy súlyozott gráf, melynek n darab csúcsa a bemeneteket reprezentálja, k darab csúcsa pedig a kimeneteket. Az n és k csúcsok halmazát rendre input és output layernek (rétegnek) nevezzük, azonban a maradék csúcsok több kisebb halmazra bonthatóak, az úgynevezett hidden (rejtett) rétegekre. 

Egy egyszerű neurális háló tehát L darab layerből áll, melyek közül az input és output layerek speciális funkciót látnak el. A rétegeknek van egy rögzített sorrendje, amely egy neurális hálónak fix tulajdonsága, sosem változik meg. Sorrendben az első réteg az input layer, amit néhány rejtett réteg követ, majd egy output layer zárja a listát.

Minden réteget neuronok egy halmaza alkot. Ezek a gráf csúcsai. Az élek kizárólag a szomszédos layerek neuronjai között futhatnak. A gráf összes éléhez tartozik továbbá egy-egy valós számértékű súly.

A neuronoknak van egy úgynevezett aktivációs értékük, amely minden pillanatban az az érték, amelyet utoljára - manuálisan vagy automatikusan - beállítottunk neki. Egy neuron aktivációs értéket úgy határozzuk meg, hogy vesszük a neuron rétegét megelőző layer azon neuronjait, amelyekkel össze van kötve, majd vesszük ezeknek a lineáris kombinációját a hozzájuk tartozó élek súlyaival. Ehhez még hozzáadunk egy bias értéket, amely magához a vizsgált neuronhoz tartozik és szintén egy 0 és 1 közé eső skalár.
Ezután még az így kapott összegre alkalmazunk egy úgynevezett aktivációs függvényt, amely ugyancsak a neuronhoz tartozik (bár általában egy rétegben minden ez neuronra azonos).

\begin{figure}[h!]
  \includegraphics[width=\linewidth]{neuron.png}
  \caption{Egy neuron}
\end{figure}

Az aktivációs függvény rendeltetése, hogy a lineáris kombináció és a bias hozzáadása után keletkezett - várhatóan 1-nél nagyobb - számot visszaskálázzuk a [0-1] intervallumba.
Többféle aktivációs függvény is használhatunk, leggyakoribbak a:

\begin{itemize}  
	\item Sigmoid: $f(x) = \frac{e^x}{e^x+1}$ 
	\item ReLU : $f(x) = max(0,x)$
\end{itemize}

Az input neuronoknál nem tudjuk kiszámolni a lineáris kombinációt, mivel nincs előző réteg, az ő esetükben egyenesen az aktivációs értéket állítjuk be inputnak. Miután minden input neuron aktivációs értékét beállítottuk látható, hogy az egész hálón végig fog menni egy változás, mert iteratívan minden réteget megelőző réteg megváltozik, tehát a maga a réteg is megváltozik.

Arra, hogy a neurális háló súlyai és biasei milyen értékek kezdetben, több lehetőség is van:
\begin{itemize}  
	\item mindegyik 0 vagy mindegyik súly 1
	\item valamilyen eloszlásból generált random számok
	\item fájlból betöltött értékek (tanítás folytatása esetén)
\end{itemize}

Ha tehát megadunk egy inputot, akkor az meghatározza az output layer neuronjainak konkrét aktivációs értékeit. Ez az érték lista a függvény kimenete. Érezhető, hogy ha egyszerűen tetszőleges inputokra rá eresztjük ezt a kalkulációt, akkor az nagyjából random outputokat fog generálni. A tanítás alapú mesterséges intelligencia algoritmusoknál azonban rendelkezésünkre állnak helyes input-output párok egy listája, vagyis egy inputhoz már kétféle outputot is tudunk mondani: az adathalmaz outputját, illetve ezt a randomnak tűnő outputot amit a háló generál.

\begin{figure}[h!]
  \includegraphics[width=\linewidth]{kapott_vart.png}
  \caption{A kapott érték nem egyezik meg a várttal}
\end{figure}

Ha a hálónk, nem képes a várt kimenetet megközelíteni, akkor a mi megközelítésünkben hibázik. Olyan módon kell változtatni a neurális háló súlyait és biaseit, hogy ezek a hibák minimálisak legyenek, hiszen ekkor a predikció is pontos. Ennek megvalósításához elengedhetetlen, hogy a hibát számszerűsíteni tudjuk.

Azon függvényeket, melyek egy $t$ várt és egy $x$ kapott vektorhoz kiszámítanak egy hibát, $\lambda(x,t)$ loss fuctionöknek nevezzük.

\subsection{Backpropagation}

Minden input-output párra meg kell változtatni a súlyokat és biaseket, mindegyiket más mértékben. Minél jobban felelős egy súly a hibáért annál jobban meg kell azt változtatni. A felelősség eldöntéséhez backpropagáljuk a hibát, vagyis a hiba vektort aktivációs értékként beállítjuk az output layernek és fordított irányban végrehajtjuk a rétegek neuronjainak aktivációját.
Például: ha a hibavektor egyik neuronja 0, tehát az adott értéket pontosan jósolta meg a háló, akkor a hozzá tartozó súlyoknak semmilyen felelőssége nincsen a hibában, ezért a 0 érték propagálódik vissza rajtuk.

\begin{figure}[h!]
  \includegraphics[width=\linewidth]{backprop_single_error.png}
  \caption{Egy neuron felelőssége a mögötte lévő rétegben}
\end{figure}

A felelősséget végül nem a súlyokon, hanem a neuronokon fogjuk realizálni. A backpropagation végeztével tehát minden neuronon lesz egy felelősség érték.

\begin{figure}[h!]
  \includegraphics[width=\linewidth]{backprop_all_error.png}
  \caption{Egy réteg teljes felelőssége a mögötte lévő rétegben}
\end{figure}

A neuronokon azonban nincs mit megváltoztatni, csak a súlyok és biasek változtatása lehetséges. A súlyok megváltoztatására a képlet az alábbi:
$$ w_{new} =  w_{old} + b_{error} * a_{activation} * lr $$
ahol $w$-vel jelölt értékekkel, a megváltoztatás előtti, illetve utáni súlyokat jelöljük, $a$ az - irányított - él kezdőpontja, $b$ pedig a végpontja, $lr$ pedig a tanulás sebessége.

A backpropagationt több input-output párra is el kell végezzük, ekkor visszünk be információt a rendszerbe arról, hogy mely párokat tartunk helyesnek, és végül az algoritmus ebből fogja ezt a véges tudását általánosítani bármilyen - korábban nem látott - inputra. Azt a folyamatot amikor az összes kívánt párra lefuttatjuk a backpropagationt $epochnak$ nevezzük. 

Ha általában meg szeretnénk mérni azt, hogy a jelenlegi háló milyen hatásfokkal tud perdiktálni, akkor nem elég a tanítás közben megnézni, mekkora hibákat backpropagál program, mivel az egész háló arra van optimalizálva, hogy erre a konkrét tanító adathalmazra hatékonyan működjön. Éppen ezért a tanítás előtt két részre osztjuk az adathalmazt:

\begin{itemize}  
	\item tanító adatok (train set): amelyeken végrehajtuk a backpropagationt, 1 epochban minden adatra egyszer
	\item validációs adatok (validation set): minden epoch végén leteszteljük ezekre az adatokra, hogy mekkora hiba keletkezik, backpropagationt viszont nem futtatunk a hibákon
\end{itemize}

A backpropagationt a gyakorlatban nem szokás minden egyes tanító adatpár után lefuttatni, hanem egyszerre több adatnak az aggregált hibáját vezetjük vissza a súlyok megváltoztatásához. Ez párhuzamosításra ad lehetőséget, amely nagy mértékben tudja csökkenteni a futásidőt.

Az epoch szám megválasztásánál figyelni kell arra, hogy a hálót ne tanítsuk túl (overfitting). vagyis ne specialázájuk rá túl nagy mértékben a tanító adathalmazra, valamint ennek ellenkezője, az alul tanítás (underfitting) is pontatlanságot eredményez, amikor egyszerűen a háló még a tanító adatokra sem képes megfelelő pontossággal prediktálni.

\subsection{Konvolúciós hálók}

A konvolúciós hálókat a legtöbb esetben kép klasszifikációra használják, vagyis háló feladata az, hogy az inputjára kapott képről (egy előre meghatározott listából) elndöntse, hogy  mi szerepel rajta (például: kutya vagy macska). Ha $n$ féle osztályt klasszifikál a háló, akkor $n$ darab output neuronja van, a kép inputra adott válasza pedig egy $n$ elemű eloszlás arról, hogy a háló jelenlegi tudása szerint melyik osztály előfordulását milyen valószínűnek tartja.

A korábbiakban általánosságban tárgyaltuk a neruális hálókat, csak annyit állítottunk, hogy a szomszédos rétegek neuronjai vannak egymással kapcsolatan. Általános esetben egy réteg minden neuronja összeköttetésben áll az őt megelőző réteg mindenen neuronjával. Az ilyen réteget teljes-összeköttetésű rétegnek nevezik, a kizárólag ilyeneket tartalmazó hálók pedig sűrű (dense) hálók.

Kép inputok esetén a sűrű hálók, azonban nem használják ki a képben rejlő azon információt, hogy bizonyos pixelek közelebb vannak egymáshoz mint más pixelek, hiszen az input neuronok szerepe a teljes-összeköttetés miatt felcserélhető. Egy sűrű hálónak ebben az esetben tehát még azt is meg kell tanulnia, hogy mit jelent képnek lenni (összefüggő egyszínű tartományok az egymáshoz közeli pixeleknél, néha éles színváltás).

Ezt a problémát oldják meg a konvolóciós hálók úgy, hogy nem kizálagosan teljes-összeköttetésű rétegeket alkalmaznak, hanem egyéb speciális rétegeket is:

\begin{itemize}
  \item Konvolúciós réteg
  \item Pooling réteg
\end{itemize}

A konvolúciós hálók általában úgy vannak felépítve, hogy néhány konvolúciós rétegenként tartalmaznak egy pooling réteget, majd a kimeneti réteg - amikor már csak egy 1 dimenziós vektort várunk - egy teljes-összeköttetésű réteg.

A súlyok javítására ugyanúgy a backpropagation algoritmus használjuk.

\subsubsection{Konvolúciós réteg}

A konvolúciós hálók rétegeit elsősorban három dimenziósnak képzeljük el. Színes kép esetén például az input réteg három dimenziója: a kép magassága, szélessége és a pixelek színe. A réteg példában szereplő szín dimenzióját a réteg mélységének nevezzük.

A konvolúciós rétegek $K$ darab szűrőből (filterből) állnak, melyek $n\times n$-es súly mátrixok, ahol $n$ általában egy kis egész szám, például 3, paraméter. Maga a réteg pontosan $K$ mélységű, minden filter egy alréteget határoz meg.

A réteg neuronjainak aktivációját a következőképpen számoljuk ki: az előző rétegnek vesszük az összes $n\times n\times d$-s ablakát, ahol $d$ az előző réteg mélységét, az ablak pedig egy neuronokból képezhető téglatestet jelöl, első réteg esetén például a középső $3\times3$ pixelt és színeiket. Minden ilyen ablak $K$ darab neuron aktivációját határozza meg, tehát minden filterhez egyet. A $k$. filter esetén az aktiváció az ablak neuron aktivációinak a filter súlyaival vett lineáris kombinációja lesz. Ezt hívják konvolúciónak.

\begin{figure}[h!]
\begin{center}
  \includegraphics[width=0.5\linewidth]{depthcol.jpg}
  \caption{Egy konvolúciós réteg összes neuronja egy ablak pozícióhoz}
\end{center}
\end{figure}

Egy konvolúciós rétegnek vannak egyéb - kevésbé fontos - paraméterei is a filterek száma és az ablakméret mellett, például az hogy az ablak eltolásánál hány neuronnyit lépünk (stride) vagy az ablak viselkedése a neuronmátrix határán (padding).

\subsubsection{Pooling réteg}

Pooling rétegeket általában két konvolúciós réteg között szoktak használni a későbbi rétegek méretének csökkentése céljából. Ez csökkenti a súlyok számát, így gyorsítja a futást, azoban ez a múvelet infromáció veszteséggel jár. Ha a képes analógiát tekintjuk, ez egy veszteséges képméret csökkentésnek felel meg (lásd \ref{pool} ábra). A konvolóciós rétegeknél is (megfelelő padding típussal) csökken a méret, a pooling réteg esetében általában nagyságrendileg felezni szokás a szélességet és magasságot.

\begin{figure}[h!]
\begin{center}
  \label{pool}
  \includegraphics[width=0.5\linewidth]{pool.jpeg}
  \caption{Egy konvolúciós réteg összes neuronja egy ablak pozícióhoz}
\end{center}
\end{figure}

\subsection{Autoencoderek}

Az autoencoderek olyan speciális neurális hálók, amelyek képesek az input rétegükre érkező adatot egy rövid kóddá tömöríteni (encode), majd abból a kódból az eredeti inputot visszaállítani (decode). Ezt a két funkciót egyetlen hálóval valósítjuk meg, azonban a hálón belül jól elkülönül a két rész: a rétegek első fele végzi el a kódolást, a második fele pedig a dekódolást. Az encodernek folyamatosan csökkennek, a decodernek pedig növekszenek a rétegeik mérete, ahogy ezt a \ref{AE_arch2} ábra mutatja. Az encode-olás és decode-olás egymás utáni elvégzése elméletben egy identitás. A tanítás során ezért a kimeneten várt érték maga az input, nincs szükségünk adatpárokra, pusztán kódolandó adatokra. A háló tanításásához szükséges hibát tehát a rekonstrukcó hatásfoka fogja meghatározni.

A háló "közepén" elhelyezkedő legkisebb méretű réteget reprezentációs vagy látens rétegnek nevezzük, a kiértékelés során az itt megjelenő aktivációs értékek vektora jelenti a kódot. Ez a kód valóban az inputra érkező adatot reprezentálja, hiszen az adatból (az encoderrel) elő tudjuk állítani a kódot, a kódból pedig (a decoderrel) vissza tudjuk állítani az adatot.

A \ref{AE_arch2} ábrán látható input egy kép, az MNIST adathalmazból, mely nagy mennyiségű alacsony felbontású képeket tartalmaz rajzolt számjegyekről. Látható, hogy a (már előzőleg betanított) háló kis hibával, de képes rekonstruálni a képet (hiszen a kimenet valóban hasonlít a bemenetre). A kép vektor reprezentációja (kód) a középső két neuronjának aktivációjából olvasható ki. A háló (pontosabban a decoder) tehát képes volt 2 számból előállítani $784$ számot (egy $28\times 28$ pixel méretű képet), vagyis az autoencoderek nem csak kódolásra, hanem tömörítésre is alkalmasak.

\begin{figure}[h!]
\begin{center}
  \label{AE_arch2}
  \includegraphics[width=\linewidth]{AE_arch2.png}
  \caption{Autoencoder}
\end{center}
\end{figure}

\subsection{Generatív modellezés}

A generatív modellezés célja, hogy egy adott (felirat mentes) adathalmazból tanulva új adatokat tudjuk generálni ugyanazon eloszlásból, mint amelyből a tanító adatok származnak. Például ha macskákról készült képekkel tanítjuk a modellt, akkor az képes lesz új, korábban nem látott macskákat tartalmazó képeket generálni.

A gyakorlatban használt két legfőbb generatív modell architektúra:
\begin{itemize}
  \item Generative Adversarial Network (GAN)
  \item Variational AutoEncoder (VAE)
\end{itemize}

Az eddigi eredmények azt mutatják, hogy a GAN-ok általában jobb eredményeket adnak, mint a VAE-k, azonban a dolgozatban a VAE-ket fogjuk mélyebben vizsgálni, a GAN-ok csak említés szintjén szerepelnek.

\subsubsection{Generative Adversarial Network (GAN)}

A GAN-ok, tehát ahogy korábban már említettük, olyan neurális hálózatok, melyek képesek az input adathalmaz eloszlásából újabb, véletlenszerű adatokat generálni. Ezt - az autoencoderekhez hasonóan - két egymástól funkcionálisan eltérő neurális háló együttes használatával érjük el a GAN esetében. A két hálót \textit{generátornak} és \textit{diszkriminátornak} nevezzük.

A generátor feladata, hogy egy véletlenszerű kódból előállítson egy adatot az adathalmaz eloszlásából. Az háló inputja tehát egy véletlen zaj, azaz minden input neuronnak egy véletlen értéket adunk meg mind tanításkor, mind kiértékeléskor. Már csak annyi a kérdés, hogy hogyan tudjuk eldönteni az output rétegén megjelenő generált adatról, hogy valóban a kívánt adathalmazból származik-e. Ezen eldöntendő kérdés megválaszolására használjuk a diszkriminátort. A generátor súlyait ezért annak függvényében változtatjuk, hogy a diszkriminátor milyen mértékben tartja valódinak az adatot.

A diszkriminátor feladata tehát, hogy egy adatról eldöntse, valóban az adathalmazból származik-e. Az inputja egy adat, az outpuja pedig mindössze két neuronból áll: egy igen és egy nem választ jelző neuronból. A diszkriminátornak felváltva adunk adatokat az adathalmazból és a generátor kimenetéről, mindkét esetben tudjuk, hogy minek kell lennie a várt eredménynek, így idővel megtanulja megkülönböztetni egymástól a valós és generált adatokat. 

A két hálót tehát párhuzamosan tanítjuk, a generátor folyamatosan megtanul adatokat generálni az adathalmaz eloszlásából, a diszkriminátor pedig folyamatosan megtanulja "leleplezni" a generátort. A modell alapvető felépítését a \ref{gan} ábra személteti.

\begin{figure}[h!]
\begin{center}
  \label{gan}
  \includegraphics[width=\linewidth]{gan.png}
  \caption{Generative Adversarial Network}
\end{center}
\end{figure}

Az utóbbi néhány évben nagy fejlődésen mentek kereszült a GAN-ok. A \ref{gan_progress} ábrán látható képeket a Celeba adathalmaz (lásd. később) eloszlásából generálták GAN architektúrájú hálókkal. Jól látszik, hogy az elmúlt 5 évben mekkora fejlődésen ment keresztül a szakirány. 

\begin{figure}[h!]
\begin{center}
  \label{gan_progress}
  \includegraphics[width=\linewidth]{gan_progress.jpg}
  \caption{GAN modellek fejlődése az elmúlt 4 évben}
\end{center}
\end{figure}

\subsection{Variational Autoencoder (VAE)}

A variational autoencoderek a GAN-okhoz hasonlóan generatív modellek, azonban a felépítésük - ahogy a nevüben is szerepel -  nagyban hasonlítanak a hagyományos autoencoderekhez.
Első ránézésre az autoencoderek is alkalmasnak tűnnek a generálásra, hiszen a decoder komponensük képes egy látens vektorból előállítani egy adatod, ezért ha egy korábban ismeretlen látens vekorra futtatjuk a decodert, akkor egy új, de mégis hiteles adatot kapnánk. Ez azonban nem működik, mert az autoencoder látens terében a gyakorlatban bizonyos térrészek teljesen érintetlenül maradnak, így a háló nincs felkészülve arra, hogy a látens tér azon szegmenséból generáljon (decode-oljon) adatot.

Ezt a probmémát oldja meg a VAE azzal a trükkel, hogy az encodere nem egy konkrét látens vektort állít elő, mint az adat reprezentációs alakja, hanem egy normáls eloszlást. A variational autoencodereknél valójában ez az eloszlás reprezeltálja az adatot, ami két vektort jelent: egy várható érték és szórás vektort, melyek koordinátánként értendőek a látens vektorra nézve. A tanítás során tehát a decoder (amit a VAE esetében már generátornak is nevezhetünk) inputja egy véletlenszerű vektor ebből az eloszlásból (lásd \ref{vae} ábra).

\begin{figure}[h!]
\begin{center}
  \label{vae}
  \includegraphics[width=\linewidth]{vae.jpg}
  \caption{GAN modellek fejlődése az elmúlt 4 évben}
\end{center}
\end{figure}

A VAE-k másik fő eltérése a hagyományos autoencoderektől, hogy a backprogpagarion során nem pusztán a rekonstrukció sikeressége szerint javítják a háló súlyait, hanem az KL (Kullback–Leibler) divergencia függvényében is. A KL divergencia nem a kimeneten kapott adatot minősíti, hanem azt, hogy a létrejött eloszlások hogyan helyezkednek el a látens térben. Röviden: az olyan eloszlás halmazokat jutalmazza, amelyek egyenletesen fedik le a látens teret. A tanítás során ezért az adathalmaz adataihoz párosítható látens reprezentációk az egész teret lefedik.

A tanítás végeztével tehát a látens térből bárhonnan választhatunk véletlen vektorokat, azokból valósághű adatotokat fog generálni a decoder.

\subsection{Módosított variational autoencoderek ($\beta$VAE)}

A $\beta$VAE-k nagyon hasonlóak a rendes VAE hálókhoz, mindössze a loss functionben különböznek egymástól. A VAE-k esetében ez a veszteség két komponensből áll: a rekonstrukciós veszteségből és a KL divergenciából. A $\beta$VAE-ben szereplő béta minössze egy konstanst jelent, amely azt határozza meg, hogy a teljes veszteség kiszámításnál a két komponenst milyen súllyal számoljuk, egészen konkrétan a rekonstrukciós veszteségnek mindig $1$, a KL divergenciának pedig $\beta$ a súlya. A rendes VAE hálók tehát valójában $\beta=1$ paraméterű $\beta$VAE-k.

A rekonstrukciós veszteség és a KL divergencia két egymásnak ellentétes erőhatásként képzelhető el a látens tér pontjaira nézve, ahol a rekonstrukciós veszteség célja az, hogy a pontok minél inkább távolodjanak az origótól (ezért egymástól is), így könnyebben megkülönböztethetőek legyenek a generálásnál, a KL divergencia pedig éppen az origó felé mozgatja azokat, hogy a térben kevés olyan térrész legyen ami betöltetlen. A $\beta$ paraméter ennek a két tendenciának az egymáshoz viszonított erősségét szabályozza.

A gyakorlati tapasztalat azt mutatja, hogy az $1$-től eltérő $\beta$-k sok esetben javítanak az eredményeken.

\section{Feladat}

A dolgozatban különböző VAE implementációkat tesztelünk egy bizonyos Dsprite adathalmazon (lásd később). Megvizsgáljuk, hogy a háló különböző paramétereinek változtatása, mint például (learning rate, látens tér dimenziója, béta hiperparaméte) milyen hatással van a loss functionökre és a rekonstrukcióra.

Továbbá megnézzük azt is, hogy az adathalmazon elvégzett transzformációk (például a képen lévő alakzat eltolása, forgatása) hogyan képződnek le a látens térbe. Ugyanis, ha tudjuk a látenstérben és az adathalmazon vett transzformációk egymáshoz való viszonyát, akkor képesek vagyunk kondicionálisan is adatot generálni. Ez azt jelenti, hogy nem csak véletlen adatokat tudunk generálni a VAE hálóval, hanem generálandó adat bizonyos tulajdonságait explicit meg tudjuk adni a generáláskor (például, hogy hol helyezkedjen el a képen belül az alakzat).

\subsection{Dsprite adathalmaz}

A kísérletekhez használt adatokat a Dsprite adathalmazból vesszük, mely a GitHubon szabadon elérhető, letölthető.

Az Dsprite egy úgynevezett szintetikus adathalmaz, ami azt jelenti, hogy az adatai nem a valóságból származnak, hanem külön erre a célra lettek legenerálva egy másik programmal. Maguk az adatok egészen pontosan $64\times64$ pixel felbontású képek, melyeken külöböző alakzatok sokféle elforgatásban, eltolásban és méretben szerepelnek. Az adathalmaz továbbá címkézve is van, tehát minden képhez tartozik a rajta szereplő alakzat pontos leírása, egészen konkrétan:

\begin{itemize}
  \item típus: az alakzat formája ($3$ féle lehet: négyzet, ellipszis, szív)
  \item méret: az alakzat mérete ($6$ féle lehet)
  \item orientáció: az alakzat elforgatottságának mértéke ($40$ féle lehet)
  \item pozíció: az alakzat elhelyezkedése x és y irányban ($32\times32$ féle lehet)
\end{itemize}

A Dsprite ezen attribútumok összes lehetséges kombinációját tartalmazza, tehát az adathalmaz pontosan
$$ 3\cdot6\cdot40\cdot32\cdot32= 737280$$
darab adatot tartalmaz. A \ref{samples} táblázatban látható néhány minta az adatok közül.

\begin{figure}[h!]
\begin{center}
\label{samples}
\begin{tabular}{| m{2.5cm} | m{2.5cm} | m{2cm}  | m{2cm} |  m{2.5cm} |}
\hline
Kép/Adat & Alak & Méret & Orientáció & Pozíció $(x,y)$ \\
\hline\hline
\includegraphics{sample00000.png} & $1$ (Négyzet) & $1$ & $1$ & $(1,1)$ \\
\hline
\includegraphics{sample0051616.png} & $1$ (Négyzet) & $1$ & $6$ & $(17,17)$ \\
\hline
\includegraphics{sample1001616.png} & $2$ (Ellipszis) & $1$ & $1$ & $(17,17)$ \\
\hline
\includegraphics{sample1501616.png} & $2$ (Ellipszis) & $6$ & $1$ & $(17,17)$ \\
\hline
\includegraphics{sample2003131.png} & $3$ (Szív) & $1$ & $1$ & $(32,32)$ \\
\hline
\end{tabular}
\end{center}
\caption{GAN modellek fejlődése az elmúlt 4 évben}
\end{figure}

Ezen a ponton felmerülhet a kérdés, hogy miért foglalkozunk azzal, hogy az adatok mellé milyen extra informácó (címke) párosul az adathalmazban, amikor korábban már láttuk, hogy az autoencoderek és VAE-k esetén pusztán a címkézetlen adatokat használjuk. Valóban, sem a hálók tanításhoz, sem a kiértékelésükhöz nem fogjuk használni ezeket a csatolt tulajdonságokat, azonban nem állunk meg azon a ponton ahol a VAE. A dolgozatban nem csupán adatokat generálunk a megadott adahalmaz eloszlásából, hanem szeretnénk megérteni a létrejött látens teret.

\section{Megvalósítás}

\subsection{Az alkalmazott szoftverarchitektúra ismertetése}

A matemaikai és egyéb kutatási irányok esetén többnyire a Python programnyelvet szokás használni, így van ez a neurális hálóknál is. Gyakorlati alkalmazás (például egy Androidos arcfelismerő) esetén természetesen már más nyelveket is szoktak használni, a mi programunknak azonban nincsen semmilyen közvetlen gyakorlati haszna, pusztán kutatási célokat szolgál.

A szoftverünkben felhasznált neurális hálókat kezelő könyvtárak a TensorFlow és a Keras, melyek nem egymástól függetlenül, hanem egymás segítve működnek. Ezek egyszerű, magasszintű hozzáférést biztosítanak a hálókhoz, ezért nem kell külön megírni például a backprogpagation algoritmust, csak meghívni az alábbi metódust:

%\lstset{language=Python}
\begin{python}
model.fit(train_data,train_data)
\end{python}

Természetesen megadható számos paraméter a tanító adatpárokon kívül, például a epoch szám, vagy a batchek mérete. Észrevehetjük, hogy a fenti függvényhívás mindkét paramétere a tanító halmaz. Ennek az a magyarázata, hogy a hálónk egy (variational) autoencoder, tehát a tanítás célja a rekonstrukció, azaz akkor tekintjük az eredményt a legjobbnak, ha a kimeneten ugyanazt látjuk mint a bemeneten. Egy klasszifikáló háló esetében például az első paraméter az adatok, a második pedig a hozzátartozó címkék lennének.

A keretrendszer úgy van kialakítva, hogy a hálók alapvető paramétereit (epoch szám, batch méret, látens tér mérete, $\dots$) a programkód változtatása nélkül, egy külön (.ini kiterjesztésű) fájlból változtathatjuk. Egészen konkrétan a program kiinduló pontját jelentő generative.py programot egyetlen argumentummal, egy .ini fájlhoz vezető úttal kell meghívni. Alább látható egy példa egy adott paraméterezésű háló betanítására:


\lstset{language=sh}
\begin{python}
python generative.py vae.ini
\end{python}

Itt tehát a vae.ini fájl minden információt tartalmaz a tanítandó hálóról, az adathalmaz is linkelve van benne. Egy ilyen paraméter fájl ezért pontosan leír egy kísérletet, utólag is megtekinthető (készül róla egy másolat is, ami a betanított háló fájljai mellé kerül). Itt látható példa egy kísérletre (.ini fájlra):

\begin{python}
activation	relu
batch_size	50
color	False
dataset	dsprite
encoder	conv_deconv
encoder_conv_channels   32,32,32
encoder_use_bn	False
encoder_wd	0.0
frequency	2
generator	conv_deconv
generator_conv_channels   32,32,32
generator_use_bn	False
generator_wd	0.0
ini_file	[]
latent_dim	10
loss_generator  mse_loss
loss_encoder    size_loss,variance_loss
lr	0.0003
memory_share	0.45
metrics	mse_loss
model_type	autoencoder
nb_epoch	15
optimizer	adam
outdir	pictures/beta_vae_tuned
sampling	True
shape	64,64
testSize	10000
trainSize	727200
verbose	2
weights size_loss|1,variance_loss|1
\end{python}

\subsubsection{Az általunk vizsgált paraméterek ismertetése}

\begin{itemize}
  \item Tanítási idő (\pythoninline{nb_epoch}): A teljes tanító adathalmazon annyiszor tanítjuk be a hálót, amennyi az epoch-ok száma. Azért van szükség egynél több epoch-ra, mert a tanítási rátát szándékosan alacsonyra állítjuk, ezért a súlyoknak többszöri frissítésre van szükségük, hogy elérjék az optimumot.
  \item Tanítási ráta (\pythoninline{lr}): Annak a mértéke, hogy a súlyokat milyen mértékben változtatjuk ahhoz képest, hogy a backprogapation szerint mennyit kéne csökkenteni rajtuk az adott pillanatban. Ha a súly változtatás függvénye csak $ w_{new} =  w_{old} + b_{error} * a_{activation}$ lenne, akkor minden alkalommal túl sokat javítanánk a súlyokon, és a veszteség függvény értéke nagyon lassan csökkenne, vagy akár divergálna (lásd. \ref{para} ábra). Ezért kell a végén alkalmaznuk még egy $lr$ tényezőt.
  \item Súly felejtés (\pythoninline{encoder_wd} és \pythoninline{decoder_wd} (weight decay)): Minden batch-ben ezzel a mértékkel csökken az összes súly értéke. Minél nagyobb az értéke annál jobban segíti a háló regularizációját (az output mennyire érzékeny az input kismértékű megváltozására). Képek esetén például általában nem kívánatos működés, ha az input egy pixelének változása óriási változást eredményez az outputon. Külön-külön beállítható az encoderre és decoderre.
  \item $\beta$ hiperparaméter  (\pythoninline{weights}): A rekonstrukciós veszteség és KL divergencia szignifikaciájának aránya.
  \item Látens tér dimenziója (\pythoninline{latent_dim}): Az autoencoder szűk keresztmetszetének mérete. Ennyi neuronnal kell a hálónak reprezentálnia az adatokat.
  \item Háló architektúrája (\pythoninline{encoder} és \pythoninline{decoder}): A háló felépítésének típusa. Korábban tárgyaltunk erre két lehetőséget: a Súrú és Konvolúciós hálókat.
\end{itemize}

A tanítás végeztével az eredmények a keretrendszer működése szerint az .ini fájlban megadott \pythoninline{outputdir} mappába kerülnek (a fenti példában \pythoninline{pictures/beta_vae_tuned}). Ebben a directoryban a program eltárol egy "biztonsági" másolatot (\pythoninline{all_params.ini}) a .ini fájlról, hogy a futtatás után is láthassuk, milyen hálót tanítottunk. Eltáródik továbbá magának a hálónak az architektúrája az \pythoninline{encoder.json} és \pythoninline{generator.json} fájlokban (közvetlenúl ezt használja a program egy háló beolvasásához). Természetesen azt is el kell menteni, hogy mi lett a tanítás eredménye, vagyis a háló éleinek végeles súlyait. Ezeket a \pythoninline{encoder.h5} és \pythoninline{generator.h5} fájlokban tároljuk el. A háló utólagos kiértékelése/korábban ismeretlen adatokra futtatása esetén a program ezekből a fájlokból olvassa be és építi fel a hálót.

Ezeken az alapvetően szükséges fájlokon kívül a mi programunk kiment egyéb - a tanítás hatékonyságát szemléltető - kép fájlokat is. Ez néhány epochonként történik meg.

Az egyik ilyen fájl a \pythoninline{train_<epoch>.png}, amely néhány véletlenszerűen kiválasztott input/output párt rajzol ki az adott epochban. Egy ilyen kép látható a \ref{trainpng} ábrán (párosával vannak: bal oldalon az input, jobb oldalon az output).

\begin{figure}[h!]
\begin{center}
  \label{trainpng}
  \includegraphics[width=0.75\linewidth]{trainpng.png}
  \caption{Input/Output párok a tanítás során}
\end{center}
\end{figure}

A \pythoninline{test_<epoch>.png} kép nagyban hasonlít a \pythoninline{train_<epoch>.png}-hez, azonban itt csak a tesztelési adatokra adott kiemeneteket nézzük, tehát a \ref{testpng} ábrán látható adatokat nem tanítottuk meg a hálónak. Egy autoencoder esetében természetesen az elvárás éppen az, hogy olyan adatokra is képes legyen rekonstruálni, melyet korábban nem látott.

 \begin{figure}[h!]
\begin{center}
  \label{testpng}
  \includegraphics[width=0.75\linewidth]{testpng.png}
  \caption{Input/Output párok a tesztelés során}
\end{center}
\end{figure}

Az iménti két képen a háló autoencoder jellegét validálhattuk szemrevételezéssel, azonban a mi hálónk egy VAE, tehát a betanítás után képes újabb képeket generálni anélkül, hogy egy input adatot adnánk neki. Ilyen, véletlen látens vektorokból generált képeket tárolunk el a \pythoninline{random_<epoch>.png} fájlban (lásd \ref{randompng} ábra). Ezen a képen már nem párokban vannak az adatok, minden szegmens egy különböző látens vektorhoz tartozik.

\begin{figure}[h!]
\begin{center}
  \label{randompng}
  \includegraphics[width=0.75\linewidth]{randompng.png}
  \caption{Véletlenszerű látens vektorokból generált képek}
\end{center}
\end{figure}

\subsubsection{Futásidő}

A neurális hálók tanítás általánosságban véve egy lassú folyamat, azonban (a batchek használata miatt) lehetőség nyílik nagymértékű párhuzamosításra, ezért a legtöbb esetben grafikus kártyákat használnak erre a célra. A dolgozatban prezentált hálókat is mind GPU-kon, Nvidia GTX 1080 Ti kártyákon tanítottam, melyekhez az MTA Rényi Alfréd Matematikai Kutatóintézet biztosított hozzáférést. Ezek jelenleg a legnagyobb teljesítményű kereskedelmi célra gyártott grafikus kártyák közé tartoznak, azonban egy-egy háló betanítása így is megközelítőleg 1 óráig tartott rajtuk.

\subsection{Látens reprezentációk strukturáltságának számszerűsítése}

Egy variational autoencoderrel, amivel dolgozunk képesek vagyunk az input adathalmaz eloszlásából véletlen adatokat generálni. Ez a VAE hálók rendeltetés szerű működése. A mi célunk az, hogy az alakzatokon végzett transzformációk a látens térben is strukturált módon megjelenjenek.

Ezen a ponton használjuk fel az adathalmaz címkéit, tehát hogy valójában minden képhez ismerjük a rajta szereplő alakzat teljes transzforációját (pozíció, méret, forgatás). Az adathalmazból így ki tudjuk keresni az összes olyan képet, melyen egy alakzat, egy forgatásban és egy méretben szerepel, vagyis annak az alakzatnak az összes eltolását. Ezt összesen $32\times32$ adatból álló képhalmazt nevezzük $Q$-nak. A következőkben azt vizsgáljuk, hogy azon vektorok a látens térben, melyekből ezek a képek generálódnak, hogy helyezkednek el. Ez pedig, ha a genrátor jól működik, akkor ekvivalens azzal, hogy hol helyezkednek el azok a vektorok, melyeket az encoder előállít a képekből.

Az alakzatok összes lehetséges eltolását két egymástól független paraméterrel le tudjuk írni: az $x$ és $y$ pozíciókkal. Azt szeretnénk látni, hogy a látens térben is leírhatóak két koordinátával, méghozzá egy az adahalmazhoz hasonló négyzetrácsot alkotnak a látens vektorok is. Az ideális természetesen az lenne, ha a látensvektor $1-1$ koordinátája egy az egyben megfeleltethető lenne $x$-nek és $y$-nak, azonban ezt nehéz feladat kikényszeríteni. Csak annyit valósítunk meg, hogy a $Q$ halmaz egy 2 dimenziós négyzetrácsra képződjön le a látens térbe. 

Annak számszerűsítésére, hogy a látens ponthalmaz mennyire áll közel egy négyzetrácshoz két metrikát is használunk:
\begin{itemize}
  \item PCA metrika
  \item Átlagos szög metrika
  \item Párhuzamosság metrika
\end{itemize}

\subsubsection{PCA metrika}

Abból indulunk ki, hogy egy négyzetrács alapvető feltétele, hogy az
azonos koordinátájú ponjai egyenest alkotnak. A metrika valójában ezen
pontvektorok egyenességét méri, és ezek átlaga lesz a négyzetrácsra
vonatkozó általános metrika.

A PCA (Principal component analysis) procedurával (mely a sklearn Python modulban elérhető) egy pontvektort levetítünk egy 1 dimenziós egyesre, úgy hogy a vetítéssel járó információ (százalékos) $l$ veszteség minimális legyen. Ha ennek az ellentétét vesszük tehát $1-l$ mennyiséget, az azt mutatja, hogy a vetítést megelőző és azt követő ponthalmaz százalékosan mennyire hasonlít egymásra. Ez ez lesz a metrikánk az egyenesekre nézve. Ha például már a teljes pontvektor egyenezt alkot a látens térben is, akkor a metrika 1 lesz.

A négyzetrácsra vonatkozó metrika ezen egyenesek metrikáinak átlaga lesz, ahol az $x$ és $y$ orientációjú egyeneseket nem különböztetjük meg az átlagolás szemponjából.

\subsubsection{Átlagos szög metrika}

A PCA metrikához hasonlóan ezen metrika esetében is az négyzetrácsot alkotó egyeneseknek egyenes mivoltát mérjük. Az átlagos szög metrikánál egyszerűen vonalt alkotó szakaszok által bezárt szögek átlaga lesz az egyenest jellemző mérőszám. 

Az általános, négyzetrácsot leíró metrika pedig a PCA-hoz hasonlóan az egyenes metrikák átlaga lesz.

\subsubsection{Párhuzamosság metrika}

Ennél a metrikánál azt számoljuk ki, hogy a vonalak mennyire párhuzamosak egymással, ahelyett, hogy a vonalak egyenességének mértékét vizsgáljuk. 

Ez a mérőszám azonban nem bizonyult hátékonynak, mivel a korábbi két metrikától eltérő eredményeket mutat. A \ref{para} ábrán két grafikon látható, melyek a PCA metrikát vetik össze a Párhuzomosság és Átlagos szög metrikával, az ábrázolt pontok pedig a $32\times32$-es eltolás mátrix vonalait jelölik. Látszik, hogy a PCA és Átlagos szög metrikák lényegében korrelálnak, míg a PCA és Párhuzamos metrikák gyakorlatilag függetlenek.

\begin{figure}[h!]
  \label{para}
  \includegraphics[width=0.5\linewidth]{evr-vs-parallelness-ldim10.png}
  \includegraphics[width=0.5\linewidth]{evr-vs-straightness-ldim10.png}
  \caption{Bal oldalon (PCA / Párhuzamosság), jobb oldalon (PCA / Átlagos szög)}
\end{figure}


\subsection{Vizualizációs eszközök ismertetése}

Elsősorban a Matplotlib nevű Python könyvtárat használjuk a grafikonok rajzolásához, ebben a fejezetben azonban nem a grafikonok stílusára, vagy azok technikai felhasználására koncetnrálunk, hanem arra, hogy a kísérletek során milyen típusú adatokat ábrázolunk.

\subsubsection{Metrika a paraméter függvényében}

Egy kísérlet abból áll, hogy a VAE háló egy bizonyos paraméterét változtatjuk, és ennek hatását vizsgáljuk különböző metrikáinkra, veszteség függvényre és egyéb érdekes tulajdonságokra nézve. A legalapvetőbb ábrák ezért úgy néznek ki, hogy ezek közül egyet ábrázolunk az aktuális paraméter függvényében, például a PCA metrikát ábrázoljuk a tanítási idő függvényében. Egy ilyen ábra tehát egy egész kísérletet jellemez.

Ezeken az ábrákon mindig a vízszintes tengelyen szerepel az aktuális paraméter, a függőlegesen pedig az, amit éppen mérünk. Attól függően, hogy az adott paraméternél, hogyan mintavételezünk (például, ha nagyságrendenként egy-egy adatunk van), egyes grafikonok $x$ tengelye logaritmikus skálázású is lehet. A \ref{mgraph} ábrán például egy háló rekonstrukciós vesztesége látható a $\beta$ paraméter függvényében.

\begin{figure}[h!]
\begin{center}
  \label{mgraph}
  \includegraphics[width=0.75\linewidth]{vae_beta-loss.png}
  \caption{Rekonstrukciós veszteség $\beta$ függvényében}
\end{center}
\end{figure}

\subsubsection{Látens vektor halmaz}

A feladat alapján az a célunk, hogy minél struktúráltabb látens teret hozzunk létre. A nehézség azonban az, hogy a látens tér valójában sok dimenziós (annyi dimenziós, ahány neuronból áll a látens réteg), ezért önmagában nem tudjuk grafikusan ábrázolni. A legmagasabb dimenzió, ami még emberi szemmel feldolgozható a $3$, ezért $n$ dimenziós ponthalmazt $3$ dimenzióssá transzformlájuk. Nyilván ekkor, ha csak a ponthalmaz nem éppen egy háromdimenziós térre esik, ez a művelet infromáció veszteséggel jár.

A transzformációt a PCA metrikából ismert PCA algoritmussal végezzük el, amely éppen egy olyan $3$ dimenziós térre képes levetíteni a ponthalmaz, hogy a lehető legkevesebb információt veszítsünk.

Ebben a dolgozatban csak az alakzatok eltolása által kijelölt látens ponthalmazra koncentrálunk. A látens térben ezért egy bizonyos alakzat összes ($32\times32=1024$) féle elhelyezkedéséhez egy-egy látens pont tartozik. A könnyebb láthatóság érdekében ezen a grafikonon megszínezzünk a pontokat aszerint, hogy a hozzájuk tartozó alakzatnak mik a koordinátái. Egészen konkrétan minden pont színe két komponensből áll (zöld és piros), ahol a két szín az alakzat $x$ és $y$ koordinátájával arányosan változik, a tényleges színe pedig ezen két komponens keveréke. 

Egy ilyen grafikon tehát egy adott hálót és ponthalmazt jellemez. A \ref{m3D} ábrán látható példa egy ilyen grafikonra.

\begin{figure}[h!]
\begin{center}
  \label{m3D}
  \includegraphics[width=1\linewidth]{3D-example.png}
  \caption{Példa egy lántens vektor halmaz ábrázolására}
\end{center}
\end{figure}

\subsection{Kísérletek}

Azt vizsgáljuk a korábban definiált metrikáinkon keresztül, hogy az alábbi paraméterek hogyan hatnak a látens tér struktúráltságára. A struktúráltság azonban önmagában még nem eredményez egy jól működő hálót. Ahhoz, hogy egyáltalán az autoencoder szerepét betöltse a VAE, alacsonynak kell lennie a rekonstrukciós veszteségnek is (ezt egyszerűen ki tudjuk olvasni a tanítás során, nem kell új metrikát bevezetni hozzá), ezért a kísérleteknél vizsgáljuk az MSE-t (Mean Square Error) is. Ez egyszerűen az átlagos négyzetes hiba (a kapott és várt output között). Ezt mostantól hívjuk csak rekonstrukciós veszteségnek, hiszen ez az a mérőszám, ami azt írja le, hogy az autoencoder milyen mértékben tud rekonstruálni.

A kísérletekben tehát egy-egy paramétert változtatunk, a többit viszont változatlanul hagyjuk. A paraméterek alapértékei (amikor éppen nem változnak) a következőek:

\begin{itemize}
  \item Tanítási idő: 15 epoch
  \item Tanítási ráta: 0.0003
  \item Súly felejtés: 0.0
  \item $\beta$ hiperparaméter: 1
  \item Látens tér dimenziója: 10
  \item Háló architektúrája: konvolúciós
\end{itemize}

\subsubsection{Tanulási idő hatása}

Ebben a kísérletben azt vizsgáltuk, hogy a tanítási idő miként hat a háló látens terének strukturáltságára. Összesen nyolcféle hálót teszteltünk, melyek kizárólag abban különböztek, hogy hány epoch-ig tartott a tanításuk. Ezek az epoch-hok a: $3, 6, 9, 12, 15, 18, 21, 24$. Ennek megvalósítására két megoldás is lehetséges: egy háló tanítását $3$ epoch-onként félbeszakítjuk és kimentjük a releváns adatokat, vagy $8$ különböző hálót tanítunk a kezdettől, és mindegyiket más-más epoch számig futtatunk. Az alábbi adatok az utóbbi megvalósításával keletkeztek.

A \ref{epoch-loss} ábrán láthatjuk, hogy a zajtól eltekintve a $15$ és $24$ közötti epoch számok bizonyulnak a legkevesebb veszteséggel járó paraméterezésnek. Érthető az is, hogy a rövid ($3-12$ epoch-os) tanítás még nem elegendő idő a hálónak arra, hogy hatékonyan rekonstruáljon.

\begin{figure}[h!]
\begin{center}
  \label{epoch-loss}
  \includegraphics[width=0.75\linewidth]{metrics/vae_epoch-loss.png}
  \caption{Rekonstrukciós veszteség a tanítási idő függvényében}
\end{center}
\end{figure}

Vizsgáljuk meg a metrikáinkat is (lásd \ref{epoch-metrics}). A PCA metrika alapján az alacsony epoch számok eredményezik a legstruktúráltabb látens teret, azonban korábban láttuk, hogy olyankor még nagyon alacsony a rekonstrukciós képesség. Az Átlagos szög metrikát nézve pedig az látható, hogy $18$ epoch-nál volt a legkisebb az átlagos görbület ($20^\circ$), habár még $15$-nél is mindössze $27^\circ$.

\begin{figure}[h!]
  \label{epoch-metrics}
  \includegraphics[width=0.5\linewidth]{metrics/vae_epoch-pca.png}
  \includegraphics[width=0.5\linewidth]{metrics/vae_epoch-as.png}
  \caption{Bal oldalon a PCA metrika, jobb oldalon az Átlagos szög metrika}
\end{figure}

A két metrika és veszteség alakulását tekintve tehát a $18$ epochig való tanítás tűnik általánosságban a legoptimálisabbnak. A \ref{3D-epoch} ábrán látható, hogy $3$ dimenzióba vetítve, hogy néz ki egy konkrét alakzat eltolása az összes lehetséges pozícióba (összesen tehát $32\times32=1024$ darab pont van az ábrán). Az ábra a $18$ epochig futtatott hálóhoz tartozik és látszik rajta, hogy struktúráltan reprezentálja az egymáshoz képest csak egy eltolásban különböző alakzatokat. Tisztán kivehetőek szabad szemmel is az $x$ és $y$ irányú (zöld és piros) görbék, sőt még a meghajlított négyzetháló is jól látszik.

\begin{figure}[h!]
\begin{center}
  \label{3D-epoch}
  \includegraphics[width=1\linewidth]{metrics/3D-epoch18.png}
  \caption{$18$ epoch tanítás után a látens tér}
\end{center}
\end{figure}

\subsubsection{Tanulási ráta hatása}

Ebben a kísérletben a tanítási rátát változtatjuk.

\begin{figure}[h!]
  \label{para}
  \includegraphics[width=1\linewidth]{parabolas.png}
  \caption{A bal oldali ábrán egy kicsi, a jobb oldali ábrán pedig egy nagy tanítási ráta}
\end{figure}



\subsubsection{Súly felejétés mértékének hatása}

Ebben a kísérletben a súly felejtés mértékének hatását vizsgáltuk meg összesen $11$ hálóra. A paraméter tesztelt értékei: $0.0005, 0.001, 0.002, 0.003, \dots , 0.009, 0.01$.

A \ref{wd-loss} ábrán láthatjuk ismét a rekonstrukciós veszteséget, az aktuális paraméter, tehát a súly felejtés függvényében. Azt tapasztaljuk, hogy a veszteségre nincs számottevő hatása a súly felejtésnek, hiszen ezekre a paraméterekre (már pedig ennél nagyobb súly felejtést nem is szokás használni) semmilyen korreláció nincsen súly felejtés és a veszteség között. 

Érdemes megfigyelni azt is, hogy míg például az epoch számos kísérletnél a veszteségek $26$-tól $34$-ig terjedtek, addig itt mindössze egy $3.5$ pontos intervallumban helyezkednek el. Ez annyit jelent, hogy habár ránézésre véletlenszerűnek tűnnek a veszteségek, valójában arról van szó, hogy a súly felejtés lényegében nem tudja változtatni a rekonstrukciós hibát.



\begin{figure}[h!]
\begin{center}
  \label{wd-loss}
  \includegraphics[width=0.75\linewidth]{metrics/vae_wd-loss.png}
  \caption{Rekonstrukciós veszteség a súly felejtés mértékének függvényében}
\end{center}
\end{figure}

A \ref{wd-metrics} ábrára nézve a korábbiakhoz hasonlóan a PCA és Átlagos szög metrikák látszanak és hasonló a tapasztalat a veszteséghez: a látenstér struktúráltsága is független a súly felejtés mértékétől.

\begin{figure}[h!]
  \label{wd-metrics}
  \includegraphics[width=0.5\linewidth]{vae_wd-pca.png}
  \includegraphics[width=0.5\linewidth]{vae_wd-as.png}
  \caption{Bal oldalon a PCA metrika, jobb oldalon az Átlagos szög metrika}
\end{figure}

Láthatjuk tehát - legalábbis a szintetikus adathalmazunkra - , hogy a súly felejtés mértéke sem a rekonstrukcióra sem a reprezentációra nincs jól definiálható hatással.

\subsubsection{$\beta$ hiperparaméter hatása}

A hagyományos VAE hálókban nem szerepel $\beta$ paraméter, csak a kiterjesztett $\beta$VAE hálókban. Emiatt azonban nem kell teljesen megváltoztatni az alap architektúrát, mivel a VAE egy speciális változata a $\beta$VAE-nak.

A $\beta$ paraméter - amint azt már korábban említettünk - azt határozza meg, hogy a látens térben rekonstrukció origóból kifele mutató és a KL divergencia origóba befelé mutató húzó erejének mi az aránya. Ha $\beta>1$ akkor a KL divergencia, ha $\beta<1$, akkor a rekonstrukciós erő dominál ($\beta=1$, akkor pedig egy hagyományos VAE hálóról beszélünk).

Ezt a paramétert számos értékre megvizsgáltuk: $0.2$-től $2.0$-ig $0.2$-esével, majd onnantól egyre ritkábban, egészen $100$-ig. A \ref{beta-loss} ábrán a rekonstrukciós veszteséget megtekintve azt láthatjuk, hogy lényegében a $\beta$ paraméter növelésével monoton csökken, egyre kisebb mértékben és nagyjából $155$-höz konvergál. 

Ez a viselkedés teljes mértékben megfelel a várakozásoknak, hiszen $\beta$ növelése éppen azt jelenti, hogy a súlyok felülírása egyre kevésbé függ a rekonstrukciós hibától és egyre jobban függ a KL divergenciától. $\beta=40$ érték közelében éri el a háló a maximális veszteséget, tehát ez az a pont, amikor a $\beta$VAE elveszti autoecoder jellegét. Ilyen és ennél magasabb $\beta$ értékek esetén már tehát szinte egyáltalán nem képes a háló rekontrukcióra.

\begin{figure}[h!]
\begin{center}
  \label{beta-loss}
  \includegraphics[width=0.75\linewidth]{metrics/vae_beta-loss.png}
  \caption{Rekonstrukciós veszteség a $\beta$ hiperparaméter függvényében}
\end{center}
\end{figure}

Ha egy megfelelő reprezentációt keresünk, akkor természetesen az is szükséges, hogy az adatokat egyáltalán rekonstruálni képes legyen a háló. Tekintsük meg a \ref{beta-pca} ábrán a PCA metrikát minden $\beta$-ra, ahol ugyebár a magas értékeknek tulajdonítunk struktúrált látens tereket. Az ábra $\beta$-t ábrázoló tengelye logaritmikus skálázású, tehát a a valóságban a jobb felső négy pontnak nagyságrendbeli eltérése van a középső pontoktól.

A veszteség függvény beláttuk, hogy a nagyon magas $\beta$-k körében értelmetlen vizsgálódni, azonban a PCA metrika mutat két kiugró pontot, mely nagyon jól struktúrált látens teret ígér ($\beta=4$ és $\beta=8$ paraméterekre), hiszen megközelítőleg információ veszteség nélkül tudjuk a négyzetrács görbéinek pontjait $10$ dimenzióban egy egyenesre vetíteni (más szóval a PCA metrika majdnem $1$).

\begin{figure}[h!]
\begin{center}
  \label{beta-pca}
  \includegraphics[width=0.75\linewidth]{metrics/vae_beta-pca-log.png}
  \caption{PCA metrika}
\end{center}
\end{figure}

Lássuk tehát, hogy ezen két háló látens tere hogyan mutat $3$ dimenzióba vetítve (lásd \ref{3D-b4} és \ref{3D-b8} ábrák). A két ábra valójában nagyon hasonlít egymásra, a különbség nagy része az, hogy a vetítés során más lett az orientációjuk, a $\beta=4$-es ábrán azonban jobban látszik a strukrúra.

Tisztán kivehető az alakzatmátrixhoz tartozó négyzetrács, azt is látjuk, hogy az egyik tengely mentén meg van hajlítva. A másik tengely esetében viszont teljesül az az elvárásunk, hogy egy $1$ dimenziós tulajdonságot (az alakzat $x$ koordinátáját) a látens térben egy $1$ dimenziós ponthalmaz, egy egyenes reprezentáljon. A rátekintési szögtől eltekintve éppen ezt figyelhetjük meg a $\beta=8$-as ábrán is. A látens pontok tehát olyan mértékben struktúráltan helyezkednek el, hogy már egy ilyen ábra alapján is fel tudjuk ismerni az $x=4$ koordinátájú alakzatokat.

\begin{figure}[h!]
\begin{center}
  \label{3D-b4}
  \includegraphics[width=\linewidth]{metrics/3D-b4.png}
  \caption{$\beta=4$}
\end{center}
\end{figure}
\begin{figure}[h!]
\begin{center}
  \label{3D-b8}
  \includegraphics[width=\linewidth]{metrics/3D-b6.png}
  \caption{$\beta=8$}
\end{center}
\end{figure}

A reprezentáció azonban még ennél is hatékonyabb. A \ref{b4-splice} ábrán $10$ darab grafikon látható függőlegesen elhelyezve. Mindegyik grafikon egyetlen látens neuron aktivációjának változását ábrázolja annak függvényében az input kép alakzatának $x$ koordinátájának függvényében. A másik, $y$ tengely értéke konstans $16$ (tehát a középső soron iterálunk végig). Ezt a műveletet pedig - ahogy a metrikák esetében is - sokféle alakzat, méret, orientáció hármasra elvégezzünk, és az eredményeiket átlagoljuk. Ezeket látjuk a grafikonokon. 

Azt tapasztaljuk, hogy lényegében egy konkrét neuron, a $9$. (alulról a második) az egyetlen, amely aktivációja lényegesen megváltozik, amikor az $x$ koordinátát változtatjuk. Eközben az összes többi neuron változatlan aktivitást mutat, kivéve az $5.$-et amely kis mértékben, de változtatja az értékét. A $9$. neuron ráadásul nem is véletlenszerűen változik, hanem szigorúan monoton csökken, közel lineárisan az $x$ koordináta növelésével.

Ebben a példában tehát azt látjuk, hogy a látens tér $9$. koordinátája közvetlenül és önállóan reprezentálja  az alakzatok $x$ koordinátáját, lényegében fordítottan arányos vele.

\begin{figure}[h!]
\begin{center}
  \label{b4-splice}
  \includegraphics[width=0.5\linewidth]{slice_y16_all_coordsb4.png}
  \caption{látens neuronok használata, $\beta=4$}
\end{center}
\end{figure}

A \ref{beta-100} ábrán a gyakorlatban is láthatjuk, hogy a magas PCA érték ellenére, miért használhatatlan a $\beta=100$-as háló.  Ahhoz hogy ezt megértsük a PCA metrika definícióját kell használnunk. Akkor magas az érték, ha a ponthalmaz minimális veszteséggel vetíthetó egy egyenesre, a veszteség pedig a vetítések távolságának valamilye aggregált értéke. Az ábrára tekintve azonban azt látjuk, hogy a teljes ponthalmaz kiterjedése kevesebb mint $10^{-10}$ ($\beta=4$-re ez az érték $3$ körüli). Ezért bármilyen origót metsző egyenesre kis veszteséggel rávetíthetőek a pontok, hiszen már lényegében az origó pontjában tömörülnek. A PCA metrika tehát ilyen esetben garantáltan az $1$ hez fog tartani, és nem hordoz magában hasznos információt.

Az is várható természetesen, hogy a pontok $\beta=100$-nál az origóban sűrűsödnek, hiszen ekkor a KL divergencia, azaz az origóba húzó erő dominál.

Jól struktúrált látens terek esetében értelme van a látens pontokat vektorként kezelni, melyek összegezhetőek és kivonhatóak egymásból. Az \ref{interp} ábrán négy kép látható. Mindegyik kép bal felső sarkában $3$ bázis kép van kirajzolva, melyek a tényleges adathalmazból származnak. A kép maradék része egy négyzetrács, melynek bal felső ($\vec{A}$), jobb felső ($\vec{B}$) és bal alsó ($\vec{C}$) sarkaiban ezen három adatképnek a rekonstrukciója látható. A képen látható összes többi alakzat az $\vec{AB}$ és $\vec{AC}$ vektorok lineáris kombinációja $\vec{A}$-hoz hozzáadva, tehát $\vec{A}+\nu\vec{AB}+\mu\vec{AC}$ vektorú látens pontból generált kép. $\nu$ és $\mu$ a kis képek koordinátájával egyenletesen változik $0$ és $1$ között, vagyis a jobb alsó kép éppen az $\vec{A}+\vec{AB}+\vec{AC}$ koordinátájú látens pontból generálódótt. 

Ha a reprezentációnk elég struktúrált, akkor egy két látens pont közötti vektor egy transzformációként értelmezhető. Ezeken az ábrákon például az $\vec{AB}$ vektor a bal felső sarokból jobb felső sarokba eltolás transzformációját jelöli. Hasonlóan az $\vec{AC}$ vektor egy $y$ irányú transzformációt ír le. Ha pedig a két vektor összegét tekintjük, akkor jogosan várhatjuk el azt, hogy az általa leírt transzformáció a két transzformáció egymás után való elvégzését fogja jelölni. A gyakorlatban tehát egy a jobb alsó sarokban elhelyezkedő négyzetet kellene látnunk a jobb alsó képen.

A \ref{interp} ábrán látható négy rész ábra mind a fentiek alapján generálódik, mindössze annyiban különböznek, hogy más $\beta$ paraméterű ($\beta=0.2, 2,3,6$) hálók látens terét mutatják be. 

A $\beta=0.2$ esetben azt látjuk, a három input kép sikeresen rekonstruálódik, azonban a két eltolási transzformáció összege (jobb alsó kis kép) szinte megegyezik a $\vec{C}$ vektorral, ami arra utal, hogy a látens térben $A$ és $B$ túlságosan közel van egymáshoz. Azt azonban legalább megtanulta a háló, hogy az $A, B$ és $C$ látens pontok mint négyzeteket ábrázolnak, ezért a $4$. pontnak is négyzetnek kell lennie. Megfigyelhető továbbá a képek között egy olyan zóna is, ahol a háló üres képeket rajzol. Ez a $\beta$ alacsony értékéből adódik, mivel $\beta=0.2$ az origóból kifelé ható erő dominál, ezért az eloszláson távol kerülnek egymástól és a térben lehetnek olyan részek, melyek semmilyen eloszláshoz nem esnek közel. A hálónak tehát nincsen tapasztalata erről, hogy az olyan látens vektorokhoz milyen képet kell generálnia, üres képeket ad.

A $\beta=2$-es ábrán tulajdonképpen a $0.2$-es ábra problémái láthatóak, viszont az előnyei nélkül, tehát a háló nem tanulta meg a transzfomációt, viszont már az input képeket is csak részben képes rekonstruálni (jó pozícióba rajzolja, jó méretben, az alakzat körvonala azonban már nem vehető ki), az összeg helyére pedig már kifejezetten hibásan, ellipszist rajzol.

$\beta=3$-nál jön el az a pont (a vizsgált $\beta$-k közül), ahol a háló már képes a két transzformációt egymás után elvégezni, hiszen láthatjuk, hogy a jobb alsó képen valóban a jobb alsó sarokba rajzol négyzetet. Az, hogy a jó pozíció mellett még az alakzatot is helyesen négyzetnek rajzolja a véletlen műve lehet, mivel a háló már egy ennél jelentősen egyszerűbb feladatot a rekonstrukciót sem tudja elvégezni (legalábbis az alakzat formáját tekintve), mivel az $A, B$ és $C$ sarkokban köröket látunk. Ez nem feltétlenül jelenti azt, hogy a háló nehezen különbözteti meg a négyzeteket a köröktől, inkább azzal indokolható a jelenség, hogy a kör alakzat kevés információ rendelkezésre állása esetén bármilyen alakzatra jó tipp a hálónak, kevés lesz az átlagos veszteség. A kör forma tehát inkább egy megforgatott négyzet vizuális kiátlagolásából adódik, mint a kör típusú adatból.

$\beta=6$ esetén ennek a tendenciának a folyatása látható, itt már minden képen csak körök láthatóak. Transzformációs képesség szempontjából azonban minden korábbinál jobban viselkedik ez a háló, elég csak a legalsó sorra nézni, ahol szinte egy folytonos átmenet látható $x=1$-től az $x=31$ koordinátáig.

Ez a négy ábra jól mutatja a $\beta$VAE működését, látszik ahogy $\beta$ növelésével egyre csökken a rekonstrukciós képesség, azonban eközben a látens tér egyre struktúráltabb, ezért képes a transzformációkat értelmezni. Az is látható továbbá, hogy a háló többi paraméterei ilyen értékekkel semmilyen $\beta$-ra sem eredményeznek egy olyan hálózatot, amely ezt a két képességet egyszerre teljesítik.

\begin{figure}[h!]
\begin{center}
  \label{interp}
  \includegraphics[width=0.49\linewidth]{metrics/interp02.png}
  \includegraphics[width=0.49\linewidth]{metrics/interp-2.png}
  \includegraphics[width=0.49\linewidth]{metrics/interp3.png}
  \includegraphics[width=0.49\linewidth]{metrics/interp-6.png}
  \caption{Bal felső: $\beta=0.2$, jobb felső: $\beta=2$, Bal alsó: $\beta=3$, jobb alsó: $\beta=6$,}
\end{center}
\end{figure}

A $\beta$ hiperparaméterrel tehát a VAE és $\beta$VAE hálók esetében mindenképp érdemes foglalkozni, ezzel a paraméterrel lehet finomhangolni azt, hogy mennyit szeretnék elengedni a rekonstrukcióból a tér strukúráltsága értekében és fordítva. A különböző adathalmazok és felhasználási célok határozzák meg egy-egy konkrét esetben, hogy mennyire érdemes választani $\beta$-t.

\subsubsection{Látens tér dimenziójának hatása}

A látens tér dimenziójának hatását is megvizsgáltuk néhány esetre ($6, 8, 10, \dots , 20, 30, 40, 50$), melyek vetszeség függése a \ref{ldim-loss} ábrán látható. Itt gondolhatnánk azt, hogy a háló szűk keresztmetszetének (a látens tér dimenciójának) növelése javítja a rekonstrukciót, hiszen a generátornak több adat áll rendelkezésre, azonban ha csak ezt a paramétert növeljük, akkor ez éppen hogy megnehezíti a visszaállítást. Ugyannyi információból egy komplexebb hálót kell építenie.

Az veszteség diagrammon is ez látszik, tehát ahogy növeljük a látens tér dimenzióját, úgy nő a veszteség is. Az $50$-es látens dimenzió esetében a kevés veszteség vélhetően csak egy szerencsés kezdetleges súly kiosztásnak köszönhető, azonban a $6$ és $8$ méretű látens tereknél így sem ér el jobb eredményt a rekonstrukció terén. 

Nyilván a látens tér méretét nem lehet akáreddig csökkenteni a rekonstrukció javítása érdekében, hiszen egy $1$ neuronból álló szük kereszmetszet már biztosan nem tudna elkódolni egy $64\times64$-felbontású képet, legfeljebb akkor, ha minden pixel egyszerre változna. A mi adahalmazunk nem túl komplex, de ennél lényegesn bonyolultabb. Ahogy az a mellékelt címkékből is látszik, a Dsprite adathalmaz egy adatpontját $5$ számmal tudjuk leírni, mely az alakzatok $5$ független tulajdonságát jelöli.

A $\beta$ paraméteres fejezetben láttunk egy példát arra, egy folytonos tulajdonság leírására elegendő mindössze $1$ neuron a látens rétegben (lásd. \ref{b4-splice}). Ideális esetben a látens tér dimenziója és a tulajdonságok száma között bijekció van. A látens teret ezért $5-6$ dimenziósnál kevesebbnek nincs értelme választani. A \ref{b4-splice} ábrához hasonlóan $5$ tulajdonság nem fog $5$ neuronnál lényegesen többet igénybe venni, $30$ neuron esetében várhatóan $20$-at szinte egyáltalán nem használ, ezért a felesleges látens neuronok csak egy zajt, potenciális hiba lehetőséget visznek a rendszerben, ami megnehezíti a rekonstrukciót.

\begin{figure}[h!]
\begin{center}
  \label{ldim-loss}
  \includegraphics[width=0.75\linewidth]{metrics/vae_ldim-loss.png}
  \caption{Rekonstrukciós veszteség a látens tér dimenziójának függvényében}
\end{center}
\end{figure}

Nézzük meg az átlagos szög metrikánkat is ezekre az adatokra a \ref{ldim-metrics} árbán. Itt is egy romló tendenciát látunk a dimenzió növelésekor, amit éppen ugyanaz magyaráz meg mint a veszteséget: hiába növeljük a látens tér méretét, egy idő után a háló akkor sem fog több neuront igénybevenni, mert a kódolandó információ mértéke változatlan. 

\begin{figure}[h!]
\begin{center}
  \label{ldim-metrics}
  \includegraphics[width=0.75\linewidth]{metrics/vae_ldim-as.png}
  \caption{Átlagos szög metrika}
\end{center}
\end{figure}

\subsubsection{Konvolúciós architektúra hatása}

Az összes kísérlet során konvolúciós hálókat alkalmaztunk, mivel ezek a hálók kifejezetten képek feldolgozására alkalmasak, azonban elméletben használhatnánk sűrű hálókat is, ezek azonban mind a klasszifikációs, mind a rekonstrukciós feladatokban rosszabb teljesítménnyel működnek. A \ref{3D-dense} árbán látható látens tér egy sűrű hálóhoz tartozik, és azt látjuk rajta, hogy habár a hasonló pozíciójú (pont színezésű) alakzatokhoz hasonló látens koordináták tartoznak, az árba strukúráltsága nem közelíti meg a konvolúciósakét.

\begin{figure}[h!]
\begin{center}
  \label{3D-dense}
  \includegraphics[width=\linewidth]{metrics/3D-dense.png}
  \caption{Átlagos szög metrika}
\end{center}
\end{figure}

\begin{thebibliography}{9}

\bibitem{ConvNet}
CS231n Convolutional Neural Networks for Visual Recognition:
http://cs231n.github.io/convolutional-networks/

\bibitem{Dsprite}
@misc{dsprites17,
author = {Loic Matthey and Irina Higgins and Demis Hassabis and Alexander Lerchner},
title = {dSprites: Disentanglement testing Sprites dataset},
howpublished= {https://github.com/deepmind/dsprites-dataset/},
year = "2017",
}

\bibitem{Dsprite}
Variational Autoencoders Explained
http://kvfrans.com/variational-autoencoders-explained/

\bibitem{Dsprite}
Keras:
https://keras.io/

\end{thebibliography}

\end{document}

%\include{chapter1}
%\include{chapter2}
%\include{chapter3}
%\bibliography{mybib}

%----------------------------------------------------------------------------
\chapter*{Köszönetnyilvánítás}\addcontentsline{toc}{chapter}{Köszönetnyilvánítás}
%----------------------------------------------------------------------------

Köszönöm az MTA Rényi Alfréd Matematikai Kutatóintézet munkatársainak a dolgozat és az ahhoz alapul szolgáló programok megírásához nyújtott segítséget, továbbá a modellek gyors futtatását lehetővé tevő modern grafikus kártyák rendelkezésre bocsátását.

%\listoffigures\addcontentsline{toc}{chapter}{Ábrák jegyzéke}
%\listoftables\addcontentsline{toc}{chapter}{Táblázatok jegyzéke}



%\include{appendices}
%\bibliography{mybib}
\addcontentsline{toc}{chapter}{Irodalomjegyzék}
\begin{thebibliography}{99}

\bibitem{tensorflow}
Mart\'{\i}n~Abadi et~al., \textit{{TensorFlow}: Large-scale machine learning on heterogeneous systems,} (Software available from tensorflow.org),  2015.

\bibitem{bvae}
Christopher~P. Burgess, Irina Higgins, Arka Pal, Loic Matthey, Nick Watters,
  Guillaume Desjardins, and Alexander Lerchner, 
\textit{Understanding disentangling in $\beta$-vae,}
\url{arXiv:1804.03599}, 2018.

\bibitem{Keras}
Fran\c{c}ois Chollet et~al.,
\newblock \textit{Keras},
\newblock \url{https://keras.io}, 2015.

\bibitem{original-VAE}
Carl Doersch,
\newblock \textit{Tutorial on variational autoencoders},
\newblock \url{arXiv:1606.05908}, 2016.



\bibitem{VAE}
Kevin Frans,
\newblock \textit{Variational autoencoders explained},
\newblock \url{http://kvfrans.com/variational-autoencoders-explained/}, 2016,
\newblock \textsl{Let\"oltve:} 2018-10-19.

\bibitem{original-GAN}
Ian~J. Goodfellow, Jean Pouget-Abadie, Mehdi Mirza, Bing Xu, David
  Warde-Farley, Sherjil Ozair, Aaron Courville, and Yoshua Bengio,
\newblock \textit{Generative adversarial networks},
\newblock \url{arXiv:1406.2661}, 2014.

\bibitem{Autoencoder}
Seungchul Lee, \textit{Autoencoder},
\url{http://i-systems.github.io/HSE545/machin learning all/Workshop/CAE/image_files/AE_arch2.png} \newblock \textsl{Let\"oltve:} 2018-11-25.

\bibitem{ConvNet}
Andrej Karpathy,
\newblock \textit{Convolutional neural networks for visual recognition},
\newblock \url{http://cs231n.github.io/convolutional-networks/}, 2018,
\newblock \textsl{Let\"oltve:} 2018-09-30.

\bibitem{Dsprite}
Loic Matthey, Irina Higgins, Demis Hassabis, and Alexander Lerchner,
\newblock\textit{ Dsprites: Disentanglement testing sprites dataset},
\newblock \url{https://github.com/deepmind/dsprites-dataset/}, 2017.

\end{thebibliography}

\bibliographystyle{plain}
\label{page:last}
\end{document}
