%----------------------------------------------------------------------------
% Abstract in hungarian
%----------------------------------------------------------------------------
\chapter*{Kivonat}\addcontentsline{toc}{chapter}{Kivonat}

A dolgozatban bemutatjuk a manapság legsikeresebb generatív modelleket, különös hangsúlyt fektetve az autoencoder alapú VAE, illetve $\beta$-VAE architektúrákra, melyek generatív modellként is alkalmazhatóak, de a reprezentáció-tanulásban is komoly jelentőségre tettek szert. Munkánk során különböző metrikák megalkotásával formalizáljuk és számszerűsítjük, hogy ezen modellek reprezentációi mennyire felelnek meg az ilyen reprezentációk iránt támasztott informális követelményeknek. Ehhez egy szintetikus adathalmazt, a Dsprite-ot használjuk.

A vizsgálathoz szükséges programokat Kerasban, Python nyelven a numpy/scikit-learn/tensorflow/matplotlib könyvtárakra építve implementáljuk, modelljeiket pedig nagy teljesítményű grafikus kártyákon futtatjuk. 

Az így felálló keretrendszerrel számos különböző VAE alapú modellt hozunk létre és ezeket reprezentáció szempontjából, saját metrikáink alapján hasonlítjuk össze egymással. Különös hangsúlyt fektetünk a $\beta$-VAE hálók $\beta$ hiperparaméterének reprezentáció-hatására, illetve a rekonstrukciós képesség és a látens tér struktúrájának egymáshoz való viszonyára. A reprezentációk személtetéséhez többféle ábrát is generálunk minden modellhez, illetve modell halmazhoz.

\vfill

%----------------------------------------------------------------------------
% Abstract in english
%----------------------------------------------------------------------------
\chapter*{Abstract}\addcontentsline{toc}{chapter}{Abstract}

A dolgozatban bemutatjuk a manapság legsikeresebb generatív modelleket, különös hangsúlyt fektetve az autoencoder alapú VAE, illetve $\beta$-VAE architektúrákra, melyek generatív modellként is alkalmazhatóak, de a reprezentáció-tanulásban is komoly jelentőségre tettek szert. Munkánk során különböző metrikák megalkotásával formalizáláljuk és számszerűsítjük, hogy ezen modellek reprezentációi mennyire felelnek meg az ilyen reprezentációk iránt támasztott informális követelményeknek. Ehhez egy szintetikus adathalmazt, a Dsprite-ot használjuk.

A vizsgálathoz szükséges programokat Kerasban, Python nyelven a numpy/scikit-learn/tensorflow/matplotlib könyvtárakra építve implementáljuk, modelleiket pedig nagy teljesítményű grafikus kártyákon futtatjuk. 

Az így felálló keretrendszerrel számos különböző VAE alapú modellt hozunk létre és ezeket reprezentáció szempontjából, saját metrikáink alapján hasonlítjuk össze egymással. Különös hangsúlyt fektetünk a $\beta$-VAE hálók $\beta$ hiperparaméterének reprezentáció-hatására, illetve a rekonstrukciós képesség és a látens tér struktúrájának egymáshoz való viszonyára. A repezentációk személtetéséhez többféle ábrát is generálunk minden modellhez, illetve modell halmazhoz.
\vfill

